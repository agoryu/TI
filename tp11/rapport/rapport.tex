% --------------------------------------
% Document Class
% --------------------------------------
\documentclass[a4paper,11pt]{article}
% --------------------------------------



% --------------------------------------
% Use Package
% --------------------------------------


\usepackage[francais]{babel}
%\usepackage{ucs}
\usepackage[utf8]{inputenc}
\usepackage[T1]{fontenc}

\usepackage{makeidx}
\usepackage{color}
\usepackage{graphicx}
\usepackage{float}
\usepackage[hidelinks]{hyperref} 
\usepackage{geometry}
%\usepackage{lastpage}
%\usepackage{marginnote}
\usepackage{fancyhdr}
%\usepackage{titlesec}
%\usepackage{framed}
\usepackage{amsmath}
\usepackage{empheq}
\usepackage{array}
\usepackage{multicol}
\usepackage{csquotes}
%\usepackage{adjustbox}

% insert code
\usepackage{listings}

% define our color
\usepackage{xcolor}

% code color
\definecolor{ligthyellow}{RGB}{250,247,220}
\definecolor{darkblue}{RGB}{5,10,85}
\definecolor{ligthblue}{RGB}{1,147,128}
\definecolor{darkgreen}{RGB}{8,120,51}
\definecolor{darkred}{RGB}{160,0,0}

% other color
\definecolor{ivi}{RGB}{141,107,185}


\lstset{
    language=java,
    captionpos=b,
    extendedchars=true,
    frame=lines,
    numbers=left,
    numberstyle=\tiny,
    numbersep=5pt,
    keepspaces=true,
    breaklines=true,
    showspaces=false,
    showstringspaces=false,
    breakatwhitespace=false,
    stepnumber=1,
    showtabs=false,
    tabsize=3,
    basicstyle=\small\ttfamily,
    backgroundcolor=\color{ligthyellow},
    keywordstyle=\color{ligthblue},
    morekeywords={include, printf, uchar},
    identifierstyle=\color{darkblue},
    commentstyle=\color{darkgreen},
    stringstyle=\color{darkred},
}


% --------------------------------------



% --------------------------------------
% Page setting
% --------------------------------------
%\pagestyle{empty}
\setlength{\headheight}{15pt}

\setcounter{secnumdepth}{3}
\setcounter{tocdepth}{2}

\makeatletter
\@addtoreset{chapter}{part}
\makeatother 

\hypersetup{         % parametrage des hyperliens
  colorlinks=true,      % colorise les liens
  breaklinks=true,      % permet les retours à la ligne pour les liens trop longs
  urlcolor= blue,       % couleur des hyperliens
  linkcolor= black,     % couleur des liens internes aux documents (index, figures, tableaux, equations,...)
  citecolor= green      % couleur des liens vers les references bibliographiques
}

% --------------------------------------

% --------------------------------------
% Information
% --------------------------------------
\title{Compte-rendu TP11 TI : Formation des images couleur et dématriçage}
\author{Elliot VANEGUE et Gaëtan DEFLANDRE}
% --------------------------------------

\definecolor{myColor}{rgb}{0.5, 0.1, 0.75}

% --------------------------------------
% Begin content
% --------------------------------------
\begin{document}

% Set language to english
  \selectlanguage{francais}

  % Start the page counting
  \pagenumbering{arabic}

  \maketitle
  
  \mbox{}
  \newpage
  \clearpage
  
  \section{Introduction}
  Avant de pouvoir traiter une image, il faut pouvoir l'acquérir au moyen de capteur présent dans les
  caméra ou appareil photo par exemple. Dans nos appareils grand public, un capteur est composé d'une mosaîque
  de pixel. Les pixels ne peuvent représenter qu'un seul dégradé de couleur. C'est pourquoi il existe différent
  CFA\footnote{Color Filter Array} qui permettent d'obtenir une image couleur. Durant ce TP, nous allons voir
  comment est utilisé le CFA de Bayer.
  
  \section{Interprétation et simulation d'une image CFA}
  Etant donné que nous travaillons sur une image acquise avec un CFA de Bayer, nous pouvons déterminer quelle est
  la disposition des pixels de l'image. Lorsque nous regardons le coin haut gauche, qui représente un ciel bleu, nous
  constatons que le troisième pixel est plus claire que les deux premiers dans l'image CFA. Or le troisième pixel 
  a une valeur dans la composante bleu qui est plus importante que les deux premiers pixels. Donc la première configuration
  de Bayer utilisé sur notre image est la suivante : \\
  
  %TODO mettre l'image de la configuration de Bayer commençant par le bleu

  Nous allons maintenant simuler la génération d'une image CFA à partir de l'image couleur.
  %TODO
  
  \section{Dématriçage par interpolation bilinéaire}
  Nous avons dans notre cours les trois formule de dématriçage suivante :
  %TODO mettre les formules
  
  Ces formule sont équivalente au masque 
  %TODO mettre les masques
  
  %TODO tout faire en fait
\end{document}  