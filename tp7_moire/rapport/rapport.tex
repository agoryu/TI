% --------------------------------------
% Document Class
% --------------------------------------
\documentclass[a4paper,11pt]{article}
% --------------------------------------



% --------------------------------------
% Use Package
% --------------------------------------


\usepackage[francais]{babel}
%\usepackage{ucs}
\usepackage[utf8]{inputenc}
\usepackage[T1]{fontenc}

\usepackage{makeidx}
\usepackage{color}
\usepackage{graphicx}
\usepackage{float}
\usepackage[hidelinks]{hyperref} 
\usepackage{geometry}
%\usepackage{lastpage}
%\usepackage{marginnote}
\usepackage{fancyhdr}
%\usepackage{titlesec}
%\usepackage{framed}
\usepackage{amsmath}
\usepackage{empheq}
\usepackage{array}
\usepackage{multicol}
\usepackage{csquotes}
%\usepackage{adjustbox}

% insert code
\usepackage{listings}

% define our color
\usepackage{xcolor}

% code color
\definecolor{ligthyellow}{RGB}{250,247,220}
\definecolor{darkblue}{RGB}{5,10,85}
\definecolor{ligthblue}{RGB}{1,147,128}
\definecolor{darkgreen}{RGB}{8,120,51}
\definecolor{darkred}{RGB}{160,0,0}

% other color
\definecolor{ivi}{RGB}{141,107,185}


\lstset{
    language=Scilab,
    captionpos=b,
    extendedchars=true,
    frame=lines,
    numbers=left,
    numberstyle=\tiny,
    numbersep=5pt,
    keepspaces=true,
    breaklines=true,
    showspaces=false,
    showstringspaces=false,
    breakatwhitespace=false,
    stepnumber=1,
    showtabs=false,
    tabsize=3,
    basicstyle=\small\ttfamily,
    backgroundcolor=\color{ligthyellow},
    keywordstyle=\color{ligthblue},
    morekeywords={include, printf, uchar},
    identifierstyle=\color{darkblue},
    commentstyle=\color{darkgreen},
    stringstyle=\color{darkred},
}


% --------------------------------------



% --------------------------------------
% Page setting
% --------------------------------------
%\pagestyle{empty}
\setlength{\headheight}{15pt}

\setcounter{secnumdepth}{3}
\setcounter{tocdepth}{2}

\makeatletter
\@addtoreset{chapter}{part}
\makeatother 

\hypersetup{         % parametrage des hyperliens
  colorlinks=true,      % colorise les liens
  breaklinks=true,      % permet les retours à la ligne pour les liens trop longs
  urlcolor= blue,       % couleur des hyperliens
  linkcolor= black,     % couleur des liens internes aux documents (index, figures, tableaux, equations,...)
  citecolor= green      % couleur des liens vers les references bibliographiques
}

% --------------------------------------

% --------------------------------------
% Information
% --------------------------------------
\title{Compte-rendu TP7 TI : Atténuation du phénomène de Moiré}
\author{Elliot VANEGUE et Gaëtan DEFLANDRE}
% --------------------------------------

\definecolor{myColor}{rgb}{0.5, 0.1, 0.75}

% --------------------------------------
% Begin content
% --------------------------------------
\begin{document}

% Set language to english
  \selectlanguage{francais}

  % Start the page counting
  \pagenumbering{arabic}

  \maketitle
  
  \mbox{}
  \newpage
  \clearpage
  
  \section*{Introduction}
  Pour certaines répétitions de motifs dans une image, nous pouvons observer un phénomène qui déforme ce motif, selon diffèrentes conditions. 
  Ce phénomène est appelé le phénomène de Moiré. Il se produit lorsque la fréquence d’un motif est trop grande. Le but de ce TP va être de 
  comprendre et de trouver une solution pour éviter ce phénomène quelque soit l’image.\\
 
  \section{Étude des fréquences}
  Pour l’étude du phénomène de moiré, nous disposons de trois images (cf Figure 1). L’image \textit{1024\_moire\_f1} avec une grande fréquence, 
  l’image \textit{1024\_moire\_f2} avec une une fréquence inférieur à l’image $f1$ et l’image \textit{1024\_moire} qui est l’addition des images 
  $f1$ et $f2$. Nous pouvons obtenir l’image \textit{1024\_moire} dans le menu Process $\rightarrow$ Image Calculator\ldots grâce à l’opération 
  \textbf{Add} sur l’image $f1$ et $f2$.\\

  \begin{figure}[H]
   \centering
   \shortstack{\includegraphics[width=4.5cm]{../1024_moire_f1.png} \\ {\small1024\_moire\_f1.png}}
   \shortstack{\includegraphics[width=4.5cm]{../1024_moire_f2.png} \\ {\small1024\_moire\_f2.png}}
   \shortstack{\includegraphics[width=4.5cm]{../1024_moire.png} \\ {\small1024\_moire.png}}
   \caption{Image présentent pour le TP}
  \end{figure}

  La période spaciale de l’image $f1$ est de 10 pixels, ce qui donne une fréquence de $\frac{1}{10}$. Pour l’image $f2$, la période est de 100 pixels, 
  donc la fréquence est de $\frac{1}{100}$. Nous pouvons retrouver les fréquence de ces images dans le plan de Fourier, c’est la taille du rayon du cercle 
  qui a pour centre le centre du plan de Fourier. Nous remarquons également que la transformée de l’image \textit{1024\_moire} correspond aux maximums  
  des transformée de Fourier des images $f1$ et $f2$, que nous pouvons obtenir grâce à l’opérateur \textbf{Max} du menu Process $\rightarrow$ Image Calculator, 
  sur les transformées de $f1$ et $f2$.\\
  
  \section{Théorème Shannon}
  Le théorème de Shanon nous dit que : \enquote{La représentation discrète d'un signal par des échantillons régulièrement espacés exige une fréquence d'échantillonnage 
  supérieure au double de la fréquence maximale présente dans ce signal.}. Lorsque nous réduisons la taille de l’image et que nous la divisons par deux, les fréquences 
  présentent dans l’image sont multiplié par deux, car le nombre de pixels entre deux motif est divisé par deux. L’image résultante au changement d’échelle n’a aucun effet 
  de Moiré visible sur l’image. Donc l’échantillonage respecte le théorème de Shannon.\\
  
  En revanche, quand nous effectuons un échantillonage d’un facteur 1/8\up{ème}, nous pouvons observer une légère déformation des motifs de l’image. Cela s’accentue 
  d’avantage en diminuant la taille de l’image. Donc à partir d’un certain seuil, le théorème de Shannon n’est plus respecté. 
  Si on a $2*f_1=\frac{1}{10}$ alors la fréquence d'échantillonage maximum est $\frac{1}{5}$, car par la suite nous aurons $2*f_1$ < <fréquence échantillonage>.\\
% Si on a $f_1=\frac{1}{10}$ alors la fréquence d'échantillonage maximum est $2*f1$, soit $\frac{1}{5}$, car par la suite nous aurons $2*f_1$ < <fréquence échantillonage>.\\
  
  \begin{figure}[H]
   \centering
   \includegraphics[width=4cm]{../128_moire.png}
   \caption{Effet de Moiré}
  \end{figure}


  On voit que parmis les deux fréquences présentent dans l’image, seul la fréquence la plus grande subit l’effet de Moiré. Cela paraît logique avec le thèorème de Shannon, 
  puisqu’il y a une seul fréquence d’échantillonage et deux fréquences dans l’image, donc se sera toujours la fréquence la plus grande qui subira, en premier, l’effet de Moiré.\\

  Le repliement de spectre est un phénomène qui introduit des fréquence inexistante dans un signal, dans notre cas cela se produit lorsque le théorème de Shannon n’est pas respecté.
  Le problème survient lorsque la fréquence maximal d’un motif est atteinte, soit $\frac{1}{2}$.\\

  \section{Solution au phénomène de Moiré}
  
  Pour remédier au phénomène de Moiré, nous avons atténué les basses fréquences de l'image. Avec cette méthode quelque soit la fréquence
  d'échantillonage, les basses fréquences en dessous d'un certain seuil seront supprimé. Nous avons pris comme seuil 0,05 ....
  %pourquoi 0,05 si la limite c'est 1/5 on aurait du prendre 0,2
  
  \section{Conclusion}
  
\end{document}  