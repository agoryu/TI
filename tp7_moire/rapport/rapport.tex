% --------------------------------------
% Document Class
% --------------------------------------
\documentclass[a4paper,11pt]{article}
% --------------------------------------



% --------------------------------------
% Use Package
% --------------------------------------


\usepackage[francais]{babel}
%\usepackage{ucs}
\usepackage[utf8]{inputenc}
\usepackage[T1]{fontenc}

\usepackage{makeidx}
\usepackage{color}
\usepackage{graphicx}
\usepackage{float}
\usepackage[hidelinks]{hyperref} 
\usepackage{geometry}
%\usepackage{lastpage}
%\usepackage{marginnote}
\usepackage{fancyhdr}
%\usepackage{titlesec}
%\usepackage{framed}
\usepackage{amsmath}
\usepackage{empheq}
\usepackage{array}
\usepackage{multicol}
\usepackage{csquotes}
%\usepackage{adjustbox}

% insert code
\usepackage{listings}

% define our color
\usepackage{xcolor}

% code color
\definecolor{ligthyellow}{RGB}{250,247,220}
\definecolor{darkblue}{RGB}{5,10,85}
\definecolor{ligthblue}{RGB}{1,147,128}
\definecolor{darkgreen}{RGB}{8,120,51}
\definecolor{darkred}{RGB}{160,0,0}

% other color
\definecolor{ivi}{RGB}{141,107,185}


\lstset{
    language=Scilab,
    captionpos=b,
    extendedchars=true,
    frame=lines,
    numbers=left,
    numberstyle=\tiny,
    numbersep=5pt,
    keepspaces=true,
    breaklines=true,
    showspaces=false,
    showstringspaces=false,
    breakatwhitespace=false,
    stepnumber=1,
    showtabs=false,
    tabsize=3,
    basicstyle=\small\ttfamily,
    backgroundcolor=\color{ligthyellow},
    keywordstyle=\color{ligthblue},
    morekeywords={include, printf, uchar},
    identifierstyle=\color{darkblue},
    commentstyle=\color{darkgreen},
    stringstyle=\color{darkred},
}


% --------------------------------------



% --------------------------------------
% Page setting
% --------------------------------------
%\pagestyle{empty}
\setlength{\headheight}{15pt}

\setcounter{secnumdepth}{3}
\setcounter{tocdepth}{2}

\makeatletter
\@addtoreset{chapter}{part}
\makeatother 

\hypersetup{         % parametrage des hyperliens
  colorlinks=true,      % colorise les liens
  breaklinks=true,      % permet les retours à la ligne pour les liens trop longs
  urlcolor= blue,       % couleur des hyperliens
  linkcolor= black,     % couleur des liens internes aux documents (index, figures, tableaux, equations,...)
  citecolor= green      % couleur des liens vers les references bibliographiques
}

% --------------------------------------

% --------------------------------------
% Information
% --------------------------------------
\title{Compte-rendu TP7 TI : Atténuation du phénomène de moiré}
\author{Elliot VANEGUE et Gaëtan DEFLANDRE}
% --------------------------------------

\definecolor{myColor}{rgb}{0.5, 0.1, 0.75}

% --------------------------------------
% Begin content
% --------------------------------------
\begin{document}

% Set language to english
  \selectlanguage{francais}

  % Start the page counting
  \pagenumbering{arabic}

  \maketitle
  
  \mbox{}
  \newpage
  \clearpage
  
  \section*{Introduction}
  Pour certaines répétitions de motifs dans une image, nous pouvons observer un phénomène qui déforme ce motif, selon différentes conditions. 
  Ce phénomène est appelé le phénomène de moiré. Il se produit lorsque la fréquence d’un motif est trop grande. Le but de ce TP va être de 
  comprendre et de trouver une solution pour éviter ce phénomène quelle que soit l’image.\\
 
  \section{Étude des fréquences}
  Pour l’étude du phénomène de moiré, nous disposons de trois images (cf Figure 1). L’image \textit{1024\_moire\_f1} avec une fréquence élevée, 
  l’image \textit{1024\_moire\_f2} avec une fréquence inférieure à l’image $f1$ et l’image \textit{1024\_moire} qui est l’addition des images 
  $f1$ et $f2$. Nous pouvons obtenir l’image \textit{1024\_moire} dans le menu Process $\rightarrow$ Image Calculator\ldots grâce à l’opération 
  \textbf{Add} sur l’image $f1$ et $f2$.\\

  \begin{figure}[H]
   \centering
   \shortstack{\includegraphics[width=4.5cm]{../1024_moire_f1.png} \\ {\small1024\_moire\_f1.png}}
   \shortstack{\includegraphics[width=4.5cm]{../1024_moire_f2.png} \\ {\small1024\_moire\_f2.png}}
   \shortstack{\includegraphics[width=4.5cm]{../1024_moire.png} \\ {\small1024\_moire.png}}
   \caption{Image présentent pour le TP}
  \end{figure}

  La période spatiale de l’image $f1$ est de 10 pixels, ce qui donne une fréquence de $\frac{1}{10}$. Pour l’image $f2$, la période est de 100 pixels, 
  donc la fréquence est de $\frac{1}{100}$. Nous pouvons retrouver les fréquences de ces images dans le plan de Fourier, c’est la taille du rayon du cercle 
  qui a pour centre le centre du plan de Fourier. Nous remarquons également que la transformée de l’image \textit{1024\_moire} correspond aux maximums  
  des transformées de Fourier des images $f1$ et $f2$, que nous pouvons obtenir grâce à l’opérateur \textbf{Max} du menu Process $\rightarrow$ Image Calculator, 
  sur les transformées de $f1$ et $f2$.\\
  
  \section{Théorème Shannon}
  Le théorème de Shannon nous dit que : \enquote{La représentation discrète d'un signal par des échantillons régulièrement espacés exige une fréquence d'échantillonnage 
  supérieure au double de la fréquence maximale présente dans ce signal.}. Lorsque nous réduisons la taille de l’image et que nous la divisons par deux, les fréquences 
  présentes dans l’image sont multipliées par deux, car le nombre de pixels entre deux motifs est divisé par deux. L’image résultante au changement d’échelle n’a aucun effet 
  de moiré visible sur l’image. Donc l’échantillonnage respecte le théorème de Shannon.\\
  
  En revanche, quand nous effectuons un échantillonnage d’un facteur $\frac{1}{8}$\up{ème}, nous pouvons observer une légère déformation des motifs de l’image. Cela s’accentue 
  d’avantage en diminuant la taille de l’image. Donc à partir d’un certain seuil, le théorème de Shannon n’est plus respecté. 
%  Si on a $2*f_1=\frac{1}{10}$ alors la fréquence d'échantillonage maximum est $\frac{1}{5}$, car par la suite nous aurons $2*f_1$ < <fréquence échantillonage>.\\
  Si on a $f_1=\frac{1}{10}$ alors la fréquence d'échantillonnage maximum est de $2*f1$, soit $\frac{1}{5}$, car par la suite nous aurons $2*f_1$ < $frequence\ echantillonnage$.\\
  
  \begin{figure}[H]
   \centering
   \includegraphics[width=4cm]{../128_moire.png}
   \caption{Effet de moiré}
  \end{figure}


  On voit que parmi les deux fréquences présentent dans l’image, seule la fréquence la plus grande subit l’effet de moiré. Cela paraît logique avec le théorème de Shannon, 
  puisqu’il y a une seule fréquence d’échantillonnage et deux fréquences dans l’image, donc se sera toujours la fréquence la plus grande qui subira, en premier, l’effet de moiré.\\

  Le repliement de spectre est un phénomène qui introduit des fréquences inexistantes dans un signal, dans notre cas cela se produit lorsque le théorème de Shannon n’est pas respecté.
  Le problème survient lorsque la fréquence maximale d’un motif est atteinte, soit $\frac{1}{2}$.\\

  \newpage
  
  \section{Solution au phénomène de moiré}
  
  Pour remédier au phénomène de moiré, nous allons supprimer les hautes fréquences 
  de l'image. Pour cela, nous prenons une fréquence limite et nous ne conservons 
  que les fréquences inférieures à cette limite. Cette fréquence limite est 
  appelée fréquence de coupure. Dans une image ce procédé va flouter l'image. 
  Dans notre cas, les plus petites rayures, donc celle dont la fréquence est la plus grande,
  vont être supprimées. Ainsi lorsque nous allons sous-échantillonner l'image, nous n'aurons plus de phénomène
  de moiré lié aux petites rayures. Si nous sous-échantillonnons avec un très grand facteur, alors le phénomène 
  de moiré serait de nouveau visible pour les rayures $f2$.\\
  
  \begin{figure}[H]
  \centering
   \includegraphics[width=5cm]{../FFT_1024_moire.png}
   \caption{Transformée de Fourier avec filtre passe-bas}
  \end{figure}
  
  Pour ce cas d'étude, une fréquence de coupure de 0.05 est suffisamment petite 
  pour éviter le phénomène de moiré pour un sous-échantillonnage de facteur 8. 
  En effet, cette fréquence se trouve entre les deux fréquences de $f1$ et $f2$, donc 
  elle élimine seulement la fréquence $f1$, la plus grande.\\
  
  \begin{figure}[H]
  \centering
   \includegraphics[width=5cm]{../1024_moire_hf.png}
   \caption{Résultat du filtre passe-bas}
  \end{figure}
  
  Après application de la macro, nous retrouvons une image ressemblante à l'image 
  \textit{1024\_moire\_f2}. Cela est tout à fait normal étant donné que nous avons 
  retiré certaines fréquences à l'image \textit{1024\_moire} dont la fréquence $f1$.\\
  
  \begin{figure}[H]
    \centering
    \includegraphics[width=9cm]{../1024_moire_sub.png}
    \caption{Histogramme de l'erreur}
  \end{figure}
  
  Nous remarquons que l'histogramme d'erreur est gris sombre, ce qui nous montre une 
  petite différence sur la moyenne des gris des deux images. Cependant, cet histogramme 
  est pratiquement unicolore, sans grande variation de gris dans l'histogramme, nous 
  permettant de dire que les deux images sont similaires, mise à part la différence de 
  moyenne, constaté auparavant.\\
  
  \section*{Conclusion}
  Durant ce TP, nous avons vu que les fréquences dans une image peuvent jouer un rôle important. 
  Sans prendre en compte ces fréquentes, un sous-échantillonnage trop important peut 
  provoquer des repliements de spectre. Les filtres passe-bas peuvent s'avérer
  particulièrement utiles dans ce genre de situation. Ils peuvent permettre d'atténuer 
  les hautes fréquences, mais cela peut provoquer une perte d'information non 
  négligeable sur un signal.\\
  
  \newpage
  \section{Annexes}
  \begin{lstlisting}[caption=Macro d'atténuation du phénomène de moiré]
macro "filtrage passe-basFFT" {

    run("FFT");
    // recuperation de ID de la FFT
    fourier = getImageID();

    // recuperation de la taille W x H du plan de Fourier
    W = getWidth();
    H = getHeight();

    // Creation d'un masque binaire
    newImage("masque", "8-bit", W, H, 1);
    masque = getImageID();
    // fond noir
    setColor(0);
    makeRectangle (0,0, W,H);
    fill();

    fc = 0.05;

    // calcul du rayon du disque binaire à partir de la frequence de coupure fc
    // attention, la FFT etant consideree comme une image par ImageJ, le rayon doit etre calcule en pixels
    rayon = H * fc;
    print("rayon =", rayon);

    setColor(255);
    makeOval (W/2-rayon,H/2-rayon, 2*rayon,2*rayon);
    fill ();


    // Filtrage passe-bas
    selectImage(fourier);
    makeOval(W/2-rayon,H/2-rayon, 2*rayon,2*rayon);
    setColor(0);
    // Selection inverse du cercle
    run("Make Inverse");
    fill();

    // Transformee de fourier inverse pour fournir l'image filtree
    run("Inverse FFT");
}
  \end{lstlisting}


\end{document}  