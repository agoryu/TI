% --------------------------------------
% Document Class
% --------------------------------------
\documentclass[a4paper,11pt]{article}
% --------------------------------------



% --------------------------------------
% Use Package
% --------------------------------------


\usepackage[francais]{babel}
%\usepackage{ucs}
\usepackage[utf8]{inputenc}
\usepackage[T1]{fontenc}

\usepackage{makeidx}
\usepackage{color}
\usepackage{graphicx}
\usepackage{float}
\usepackage[hidelinks]{hyperref} 
\usepackage{geometry}
%\usepackage{lastpage}
%\usepackage{marginnote}
\usepackage{fancyhdr}
%\usepackage{titlesec}
%\usepackage{framed}
\usepackage{amsmath}
\usepackage{empheq}
\usepackage{array}
\usepackage{multicol}
%\usepackage{adjustbox}

% insert code
\usepackage{listings}

% define our color
\usepackage{xcolor}

% code color
\definecolor{ligthyellow}{RGB}{250,247,220}
\definecolor{darkblue}{RGB}{5,10,85}
\definecolor{ligthblue}{RGB}{1,147,128}
\definecolor{darkgreen}{RGB}{8,120,51}
\definecolor{darkred}{RGB}{160,0,0}

% other color
\definecolor{ivi}{RGB}{141,107,185}


\lstset{
    language=Scilab,
    captionpos=b,
    extendedchars=true,
    frame=lines,
    numbers=left,
    numberstyle=\tiny,
    numbersep=5pt,
    keepspaces=true,
    breaklines=true,
    showspaces=false,
    showstringspaces=false,
    breakatwhitespace=false,
    stepnumber=1,
    showtabs=false,
    tabsize=3,
    basicstyle=\small\ttfamily,
    backgroundcolor=\color{ligthyellow},
    keywordstyle=\color{ligthblue},
    morekeywords={include, printf, uchar},
    identifierstyle=\color{darkblue},
    commentstyle=\color{darkgreen},
    stringstyle=\color{darkred},
}


% --------------------------------------



% --------------------------------------
% Page setting
% --------------------------------------
%\pagestyle{empty}
\setlength{\headheight}{15pt}

\setcounter{secnumdepth}{3}
\setcounter{tocdepth}{2}

\makeatletter
\@addtoreset{chapter}{part}
\makeatother 

\hypersetup{         % parametrage des hyperliens
  colorlinks=true,      % colorise les liens
  breaklinks=true,      % permet les retours à la ligne pour les liens trop longs
  urlcolor= blue,       % couleur des hyperliens
  linkcolor= black,     % couleur des liens internes aux documents (index, figures, tableaux, equations,...)
  citecolor= green      % couleur des liens vers les references bibliographiques
}

% --------------------------------------

% --------------------------------------
% Information
% --------------------------------------
\title{Compte-rendu TP3 TI : Images discrètes}
\author{Elliot VANEGUE et Gaëtan DEFLANDRE}
% --------------------------------------

\definecolor{myColor}{rgb}{0.5, 0.1, 0.75}

% --------------------------------------
% Begin content
% --------------------------------------
\begin{document}

% Set language to english
  \selectlanguage{francais}

  % Start the page counting
  \pagenumbering{arabic}

  \maketitle
  
  \mbox{}
  \newpage
  \clearpage
  
  \section*{Introduction}
  
  \section{Composantes d'une image couleur}
  Nous allons d'abord voir comment manipulé les composantes d'une image couleur à partir de l'image de calibrage
  de couleur fournit dans le TP. On peut voir que la taille de la variable contenant l'image est de 1 440 000,
  qui correspond à $<largeur> * <hauteur> * <nombre de canaux>$.\\
  
  Pour récupérer la composante d'une couleur en image de gris, il suffit de récupérer le vecteur d'une des composantes
  et de l'afficher seul. Nous avons ainsi une matrice avec les valeurs de rouge et vu que nous ne gardons qu'une matrice,
  l'image se transforme en dégradé de gris.\\
  
  %TODO ajouter l'image avec les 3 images de gris
  
  La récupération d'une image dans un dégradé d'une seul couleur, il faut annuler les deux autres canaux en les multipliant par
  zéro.
  
  %TODO ajouter l'image avec les 3 images de couleur différente
  \section{Sur et sous-échantillonnage}
  %a revoir
  %800/72 = 11
  %600/72 = 8
  \subsection{Sous-échantillonage}
  Pour effectuer un sous-échantillonage, il faut supprimer une valeur sur N dans le vecteur de chaque canaux.
  Pour cela on récupère un vecteur avec l'ensemble des pixels de l'image, on supprime le nombre de pixel nécessaire
  et on reforme la matrice de l'image.\\
  
  %mettre le code de sous-echantillonage avec caption
  \subsection{Sur-échantillonage}
  \section{Quantification}
  \section{Repliement de spectre}
  
  \section*{Conclusion}
 
    
\end{document}  