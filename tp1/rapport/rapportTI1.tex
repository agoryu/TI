% --------------------------------------
% Document Class
% --------------------------------------
\documentclass{article}
% --------------------------------------



% --------------------------------------
% Use Package
% --------------------------------------

% french, english
\usepackage[francais]{babel}

% font, french accent
\usepackage[utf8]{inputenc} 
\usepackage[T1]{fontenc} 

% page layout
\usepackage{geometry}

% hypertext link
\usepackage[pdfpagelabels]{hyperref}

\usepackage{graphicx}
\usepackage{float}
\usepackage{verbatim}
\usepackage{fancyhdr}
\usepackage{amsmath}
\usepackage{listings}


% include pdf
\usepackage[final]{pdfpages}


% --------------------------------------



% --------------------------------------
% Page setting
% --------------------------------------
%\pagestyle{empty}
\setlength{\headheight}{15pt}

\setcounter{secnumdepth}{3}
\setcounter{tocdepth}{2}

\makeatletter
\@addtoreset{chapter}{part}
\makeatother 

\hypersetup{         % parametrage des hyperliens
  colorlinks=true,      % colorise les liens
  breaklinks=true,      % permet les retours à la ligne pour les liens trop longs
  urlcolor= blue,       % couleur des hyperliens
  linkcolor= black,     % couleur des liens internes aux documents (index, figures, tableaux, equations,...)
  citecolor= green      % couleur des liens vers les references bibliographiques
}

% --------------------------------------

% --------------------------------------
% Information
% --------------------------------------
\title{Compte-rendu TP1 TI : Sources lumineuses}
\author{Elliot VANEGUE et Gaëtan DEFLANDRE}
% --------------------------------------

\definecolor{myColor}{rgb}{0.5, 0.1, 0.75}

% --------------------------------------
% Begin content
% --------------------------------------
\begin{document}
  % for listing
  \lstset{
    language=Scilab,
    commentstyle=\color{myColor}
    }

  % Set language to english
  \selectlanguage{francais}

  % Start the page counting
  \pagenumbering{arabic}

  \maketitle
  
  \mbox{}
  \newpage
  \clearpage
  
  \section{Introduction}
  Nous allons, dans ce premier TP, étudier l'éclairement de différent type
  de source lumineuse via l'outils Scilab.
  
  \section{Eclairement d'une source ponctuelle isotrope}
  Une source ponctuelle isotrope est une surface lumineuse dont l'intensité lumineuse
  est la même dans toutes les directions. Avec cette définition nous allons pouvoir
  calculer les valeurs d'éclairement pour une surface.\\
  
  Détail du code donné en exemple.\\
  
  \begin{lstlisting}
    //Calcul la distance entre la source lumineuse et chaque
    //centimetre carre de la surface
  
    // Definition des echantillons sur un axe
    axe = [0:99] / 100 + 5e-3; //correspond a une matrice de valeur entre 0 et 1 
    // Definition des elements de surface
    x = ones (1:100)' * axe; // x et y sont des matrices
    y = axe' * ones (1:100);
    // Position de la source -> hauteur
    xs = 0.5;
    ys = 0.5;
    // Calcul de la distance
     // ensemble des valeurs qui seront represente dans le graphe
    d = sqrt ((x - xs).^2 + (y - ys).^2);
    // Trace de la fonction distance
    plot3d (axe, axe, d);
    // Visualisation sous forme d'image en niveaux de gris
    imshow (d);
  \end{lstlisting}
  
  Nous avons modifié ce code afin de pouvoir calculer la valeur d'éclairement sur la 
  surface de l'exemple du TP. Nous optenons un graphe comme celui-ci :
  
  % TODO mettre l'image du graphe

  \section{Eclairement d'une source ponctuelle lambertienne}
  %indicatrice d'émission est la sphère d'équation I = I0.cos(θ)
  
  
  \section{Eclairement d'une grille de source ponctuelle}
    
\end{document}